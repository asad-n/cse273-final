\documentclass[12pt,a4paper]{article}
\usepackage[a4paper,margin=1in]{geometry}
\usepackage[none]{hyphenat}
\usepackage[dvipsnames]{xcolor}
\usepackage{tikz}
\usetikzlibrary{calc}
\usepackage{anyfontsize}
\usepackage{multicol}
\usepackage{pgfplots}
\usepackage{float}
\usepackage{enumerate}
\usepackage{array}
\usepackage{amsmath}
\usepackage{multirow}
%\usepackage{graphicx}
%\usepackage{tabularx}
%\usepackage{titlesec}

%\titleformat{\section}[block]{\Large\bfseries\filcenter}{}{1em}{}

\newcolumntype{C}[1]{>{\centering\let\newline\\\arraybackslash\hspace{0pt}}m{#1}}

\pgfplotsset{compat=1.7}

\setlength{\parindent}{0em}
\setlength{\parskip}{2ex}
\renewcommand{\baselinestretch}{1.3}

\newcommand\linefillcentre {\begin{center}\rule{0.3\textwidth}{0.1mm}\end{center}}

\begin{document}

\begin{titlepage}
\begin{tikzpicture}[overlay,remember picture]

% Background color
\fill[
black!2]
(current page.south west) rectangle (current page.north east);

% Rectangles
\shade[
left color=Dandelion, 
right color=Dandelion!40,
transform canvas ={rotate around ={45:($(current page.north west)+(0,-6)$)}}] 
($(current page.north west)+(0,-6)$) rectangle ++(9,1.5);

\shade[
left color=lightgray,
right color=lightgray!50,
rounded corners=0.75cm,
transform canvas ={rotate around ={45:($(current page.north west)+(.5,-10)$)}}]
($(current page.north west)+(0.5,-10)$) rectangle ++(15,1.5);

\shade[
left color=lightgray,
rounded corners=0.3cm,
transform canvas ={rotate around ={45:($(current page.north west)+(.5,-10)$)}}] ($(current page.north west)+(1.5,-9.55)$) rectangle ++(7,.6);

\shade[
left color=orange!80,
right color=orange!60,
rounded corners=0.4cm,
transform canvas ={rotate around ={45:($(current page.north)+(-1.5,-3)$)}}]
($(current page.north)+(-1.5,-3)$) rectangle ++(9,0.8);

\shade[
left color=red!80,
right color=red!80,
rounded corners=0.9cm,
transform canvas ={rotate around ={45:($(current page.north)+(-3,-8)$)}}] ($(current page.north)+(-3,-8)$) rectangle ++(15,1.8);

\shade[
left color=orange,
right color=Dandelion,
rounded corners=0.9cm,
transform canvas ={rotate around ={45:($(current page.north west)+(4,-15.5)$)}}]
($(current page.north west)+(4,-15.5)$) rectangle ++(30,1.8);

\shade[
left color=RoyalBlue,
right color=Emerald,
rounded corners=0.75cm,
transform canvas ={rotate around ={45:($(current page.north west)+(13,-10)$)}}]
($(current page.north west)+(13,-10)$) rectangle ++(15,1.5);

\shade[
left color=lightgray,
rounded corners=0.3cm,
transform canvas ={rotate around ={45:($(current page.north west)+(18,-8)$)}}]
($(current page.north west)+(18,-8)$) rectangle ++(15,0.6);

\shade[
left color=lightgray,
rounded corners=0.4cm,
transform canvas ={rotate around ={45:($(current page.north west)+(19,-5.65)$)}}]
($(current page.north west)+(19,-5.65)$) rectangle ++(15,0.8);

\shade[
left color=OrangeRed,
right color=red!80,
rounded corners=0.6cm,
transform canvas ={rotate around ={45:($(current page.north west)+(20,-9)$)}}] 
($(current page.north west)+(20,-9)$) rectangle ++(14,1.2);

% Add NSU Logo
\node[align=center] at  ($(current page.center)+(0,0)$)
{
\includegraphics{nsu-logo.png}
};

\node[align=center] at ($(current page.center)+(0,-7)$) 
{
% Doc-Info
{\fontsize{40}{48} \selectfont {{-------- Final Assignment --------}}} \\[0.4cm]
{\fontsize{20}{24} \selectfont \textcolor{blue}{ \bf Theory of Computation (CSE273)}}\\[0.25cm]
{\fontsize{20}{24} \selectfont \textcolor{blue}{ \bf Sec: 2, Spring 2021}}\\[0.25cm]
{\fontsize{20}{24} \selectfont \textcolor{orange}{ Faculty Instructor: Md. Mosaddek Khan (MMK1)}}\\[0.25cm]
{ \dotfill }~\\[0.25cm]
};

\node[align=center] at ($(current page.center)+(0,-11)$) 
{
{\fontsize{20}{24} \selectfont {\bf Asaduzzaman Noor}} \\[0.25cm]
{\fontsize{20}{24} \selectfont {163 1231 042}} \\
};
\end{tikzpicture}
\end{titlepage}

% ================================ Actual Content ================================
%\pagenumbering{gobble}
{\color{PineGreen} \textbf{\textit{Q1:}} Construct a DFA with 6 states, two of which are final states. Take a string of length 7. Now, use
extended transition function to test whether the string is accepted by your constructed DFA.}

\textbf{\textit{Ans.}} Consider the DFA = (Q, $\Sigma$, $\delta$, A, F), where\\
\[ Q = \{A, B, C, D, E, F\} \]
\[ \Sigma = \{0, 1\} \]
Transition table for a DFA that has 6 states, 2 of which are final states, is given below:
\begin{table}[h!]
    \begin{center}
        \begin{tabular}{|c|c|c|}
	   %       & \multicolumn{2}{c|}{Input} \\ \hline
	    States & Input: 0 & Input: 1 \\ \hline
   $\rightarrow$A & B & C \\ \hline
		 B & C & D \\ \hline
		 C & E & F \\ \hline
		*D & B & E \\ \hline
		 E & F & C \\ \hline
		*F & F & D \\ \hline
        \end{tabular}
	\caption{DFA - Transition Table}
    \end{center}
\end{table}

Let's take a string, $w = 0100101$



{\color{PineGreen} \textbf{\textit{Q2:}} Construct a $\varepsilon$-NFA using 10 states, three of which are final states and at least two are ε -transitions.
Now, use subset construction algorithm to convert this $\varepsilon$-NFA to its equivalent DFA.}

Hello

{\color{PineGreen} \textbf{\textit{Q3:}} Consider the final DFA that you have created as an answer to Question 2. Use that DFA, and
now, use table filling algorithm to minimize it.} 

Hello

{\color{PineGreen} \textbf{\textit{Q4:}} Construct a Mealy machine with 6 states with input alphabet {0, 1, 2} and output alphabet
{a,b,c}. Now, transform this Mealy machine to its equivalent Moore machine.}

Hello

\end{document}
